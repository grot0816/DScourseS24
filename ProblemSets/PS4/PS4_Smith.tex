\documentclass{article}

% Language setting
% Replace `english' with e.g. `spanish' to change the document language
\usepackage[english]{babel}

% Set page size and margins
% Replace `letterpaper' with `a4paper' for UK/EU standard size
\usepackage[letterpaper,top=2cm,bottom=2cm,left=3cm,right=3cm,marginparwidth=1.75cm]{geometry}

% Useful packages
\usepackage{amsmath}
\usepackage{graphicx}
\usepackage[colorlinks=true, allcolors=blue]{hyperref}

\title{PS4 Smith}

\begin{document}
\maketitle

I would be interested in scraping the data from the US Bureau of Labor Statistic, focusing on women in the HVAC industry. I read an article from an HVAC publication and website

    https://www.achrnews.com/articles/153719-2023-top-women-in-hvac-list

that said only two percent of HVAC employees are women. I would love to dig deeper into that statistic to see more about locations in the US, income and education levels.
\space

\section{Question 6.7 Answer}
The df1 data frame type is a "tbl-df". The df data frame type is a "tbl-spark".
\space
\section{Question 6.8 Answer}
The df1 data frame column titles include the ".", such as "Petal.Length". However, in the spark df, they're renamed to replace the period with an underscore, such "Petal-Length". This is done to make sure the names match SQL Spark naming convention.






\end{document}